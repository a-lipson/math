\documentclass[../m134a-hw2.tex]{subfiles}

\begin{document}
\subsection*{36}
Let $f(x) = \begin{cases}
    A^2x^2, & x\leq2 \\
    (1-A)x, & x>2. \\
\end{cases}$ For what values of $A$ is $f$ continuous at 2?

The function $f$ is continuous when $\lim\limits_{x \to 2^-} f(x) = \lim\limits_{x \to 2^+} f(x) $.

Since $(1-A)x$ is polynomial, $\lim\limits_{x \to 2^-} (1-A)x = 2(1-A)$.

Since $A^2x^2$ is polynomial, $\lim\limits_{x \to 2^+} A^2x^2 = 4A^2$.

So, $f$ will be continuous when,
\begin{align*}
    2(1-A) &= 4A^2 \\
    {-4}A^2-2A+2 &= 0 \\
    2A^2+A-1 &= 0 \\
    A^2 + \frac{A}{2} - \frac{1}{2} &= 0 \\
    A^2 + \frac{2A}{4} + \frac{1}{16} &= \frac{1}{2} + \frac{1}{16} \\
    {\left(A + \frac{1}{4}\right)}^2 &= \frac{9}{16} \\
    A + \frac{1}{4} &= \pm \frac{3}{4} \\
    A &= -\frac{1}{4} \pm \frac{3}{4} \\
    A &= -1, \frac{1}{2}. \\
\end{align*}

\subsection*{53}
Suppose that the function $f$ has the property that there exists a number $B$ such that \[|f(x)-f(c)|\leq B|x-c|\] for all $x$ in the interval $(c-p,c+p)$. Prove that $f$ is continuous at $c$.

\begin{proof}
    We begin by noting that a function is continuous when $\lim\limits_{x \to c} f(x) = f(c)$.
    
    We will rewrite $x \in (c-p,c+p)$ as $|x-c|<p$.

    For consistency, we will $\delta$ such that, $p = \delta > 0$.

    For every $\epsilon>0$, there exists a $\delta>0$ such that $|x-c|<p=\delta$ implies $|f(x)-f(c)|<\epsilon$.

    Let $\epsilon>0$ be given. Define $\delta = \frac{\epsilon}{B}$.

    Assume $|x-c|<\delta$, then
    \begin{align*}
        |x-c| &< \frac{\epsilon}{B} \\
        B|x-c| &< \epsilon \\
        |f(x)-f(c)| &< \epsilon \text{ by the property of $f$.}\\
    \end{align*}

    Therefore, $\lim\limits_{x \to c} f(x) = f(c)$.

    So, the function $f$ is continuous at $c$.
\end{proof}

\end{document}