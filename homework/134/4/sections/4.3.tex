\documentclass[../hw4.tex]{subfiles}

\begin{document}

\subsection*{42}
Let $y = f(x)$ be differentiable and suppose that the graph of $f$ does not pass through the origin. The distance $D$ from the origin to a point $P(x,f(x))$ of the graph is given by
\[D = \sqrt{x^2+{[f(x)]}^2}.\]
Show that if $D$ has a local extreme value at $c$, then the line through $(0,0)$ and $(c,f(c))$ is perpendicular to the line tangent to the graph of $f$ at $(c,f(c))$.

\begin{proof}
    Since $(0,0)$ does not lie on $f$, then $D$, being the distance from $f$ to $(0,0)$, is never zero.

    Since $D$ has a local extreme at $c$, then $D'(c) = 0$ or $D'(c)$ does not exist.

    First, we find $D'$,
    \begin{align*}
        \frac{d}{dx} D &= \frac{d}{dx} \left[ \sqrt{x^2+{[f(x)]}^2} \right] \\
        &= \frac{1}{2\sqrt{x^2+{[f(x)]}^2}} \cdot \frac{d}{dx} \left[ x^2+{[f(x)]}^2 \right] \\
        &= \frac{2x+2f(x)f'(x)}{2\sqrt{x^2+{[f(x)]}^2}} \\
        &= \frac{x+ff'}{\sqrt{x^2+f^2}} \\
        &= \frac{x+ff'}{D} \\
    \end{align*}

    Then, given that $D'(c)=0$ (we assume $D$ to be differentiable at $c$ and so its critical point exists), and that $D\neq0$,
    \begin{align*}
        D'(c) = \frac{c+f(c)f'(c)}{D(c)} &= 0 \\
        c+f(c)f'(c) &= 0 \\
        f'(c) &= \frac{-c}{f(c)} \\
    \end{align*}

    Two lines are perpendicular if their slopes are negative reciprocals of each other, such that,
    \[m_1 = \frac{-1}{m_2}.\]

    The line through $(0,0)$ and $(c,f(c))$ has a slope of $\frac{f(c)}{c}$.

    The line tangent to $f$ at $(c,f(c))$ has a slope of $f'(c)$.

    Since $f'(c) = \frac{-c}{f(c)}$, which is the slope of the second line, then its negative reciprocal is $\frac{-1}{f'(c)} = \frac{f(c)}{c}$, which is the slope of the first line.

    So, the lines are perpendicular and the statement holds. 

\end{proof}


\end{document}