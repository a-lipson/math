\documentclass[../root file]{subfiles}

\begin{document}

Let the series $\sum a_k$ and $\sum b_k$ be defined such that, for all $k$, $a_k, b_k >0$.

Suppose that $\frac{a_k}{b_k}\to\infty$ as $k\to\infty$.

So, $a_k > b_k$ for all sufficiently large $k$.

(a) Since $\sum b_k$ diverges and $a_k>b_k$ for all sufficiently large $k$, then, by the Basic Comparison Theorem, $\sum a_k$ diverges as well.

(b) Since $\sum a_k$ converges and $a_k>b_k$ for all sufficiently large $k$, then, by the Basic Comparison Theorem, $\sum b_k$ must converge as well.

(c) Let $a_k = \frac{1}{k}$, the harmonic series. This series diverges.

First, Let $b_k = \frac{1}{k^2}$, the $p$-series with $p=2>1$, which converges.

Then, $\frac{a_k}{b_k}=\frac{1/k}{1/k^2}=k$.

So, $\lim\limits_{k\to\infty}\frac{a_k}{b_k}=\lim\limits_{k\to\infty}k=\infty$.

Next, let $a_k=\frac{1}{\sqrt{k}}$, the $p$-series with $p=\frac{1}{2}<1$, which diverges, and $b_k$ be the harmonic series. 

Then, $\frac{a_k}{b_k}=\frac{1/\sqrt{k}}{1/k}=\sqrt{k}$, which tends to infinity as $k$ tends to infinity.

So, if $\sum a_k$ diverges and $\frac{a_k}{b_k}\to\infty$ as $k\to\infty$, then $b_k$ can either converge or diverge.

(d) First, note that $a_k$ can diverge when $b_k$ converges as in the first example in (c). 

So, we will consider an $a_k$ and $b_k$ which converge. We will choose the 2 and 3 $p$-series respectively, 

\[a_k=\frac{1}{k^2}, \quad b_k=\frac{1}{k^3}.\]

Then, \[\frac{a_k}{b_k}=\frac{1/k^2}{1/k^3}=k,\] which we have already seen tends toward infinity as $k$ tends toward infinity.

Thus, $a_k$ can either diverge or converge when $b_k$ converges.

\end{document}