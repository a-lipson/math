\documentclass[../hw6]{subfiles}

\begin{document}

Let $y_1$ and $y_2$ be solutions to 
\begin{equation*}
    y''+py'+qy=0 \tag{*}
\end{equation*}
where $p,q$ continuous on $I=(a,b)$.

Show that if there as a point in $I$ where $y_1$ and $y_2$ are both zero or where both have maxima or minima, then $y_1$ and $y_2$ are linearly dependent. 

We will show that there exists a $c \in I$ such that, if either 
\[y_1(c)=y_2(c)=0,\tag{a}\] or 
\[{y'}_1(c)={y'}_2(c)=0\tag{a}\] and ${y''}_1(c)\neq0$ and ${y''}_2(c)\neq0$, 
then $y_1$ and $y_2$ are linearly dependent.

For a contradiction, assume that $y_1$ and $y_2$ are linearly independent solutions to (*). 

So, the Wronskian of $y_1$ and $y_2$ is non zero for all $t\in I$.
\[0 \neq W(y_1, y_2) = \begin{vmatrix}
    y_1 & y_2 \\
    y_1' & y_2'
\end{vmatrix}
= y_1 y_2' - y_1'y_2.\]

Then, for case (a), \[y_1(c)=y_2(c)=0\] implies that
\[y_1(c)y_2'(c) - y_1'(c)y_2(c)=0.\]

But, \[y_1 y_2' - y_1'y_2\neq0\] for all $t\in I$ by the assumption that the solutions $y_1$ and $y_2$ were linearly independent.

Therefore $y_1$ and $y_2$ are linearly dependent and also constant multiples of each other.

Next, for case (b), \[y_1'(c)=y_2'(c)=0\] implies that 
\[y_1(c)y_2'(c) - y_1'(c)y_2(c)=0.\]

But, this contracts the assumption of the linear independence of the solutions.

The non-zero second derivative condition was not necessary. 

So, we have shown that if there exists a $c$ such that the solutions are both zero at $c$, or both attain a critical point at $c$, then the two solutions are linearly dependent and therefore constant multiples of one another.


\end{document}
