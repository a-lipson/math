\documentclass[../hw8]{subfiles}

\begin{document}

Solve $y''+3ty+3y=0$ using series.
We begin by obtaining $y$, $y'$, and $y''$, being mindful of the starting index to avoid zero terms in the sum,
\begin{align*}
    y&=\sum\limits_{k=0}^{\infty}a_k t^k, \\
    y'&=\sum\limits_{k=1}^{\infty}k a_k t^{k-1}, \\
    y''&=\sum\limits_{k=2}^{\infty}k(k-1)a_k t^{k-2}. \\
\end{align*}

Then we use these series definitions in place of $y$ and its derivatives in the differential equation,
\[\sum\limits_{k=2}^{\infty}k(k-1)a_k t^{k-2}+3t\sum\limits_{k=1}^{\infty}k a_k t^{k-1}+3\sum\limits_{k=0}^{\infty}a_k t^k=0.\]

We make a substitution or change of variables such that all $t$ terms are to the same power,
\[\sum\limits_{k=0}^{\infty}(m+2)(m+1)a_{m+2} t^m+3\sum\limits_{m=1}^{\infty}ma_m t^m+3\sum\limits_{m=0}^{\infty}a_m t^m=0.\]

We then set all series to begin at the same index $m=1$ by extracting initial terms as well as grouping the coefficients of $t^m$,
\[2a_2+3a_0+\sum\limits_{m=1}^{\infty}\left[ (m+2)(m+1)a_{m+2}+3ma_m+3a_m \right]t^m=0.\]

Given that the sum of the series and the initial coefficients is identically zero, then both the sum initial coefficients themselves and the coefficients of the series must all be zero such such that the equation holds for all $t$.

So, 
\begin{align*}
    2a_2+3a_0&=0\\
    a_2&=-\frac{3}{2}a_0.\\
\end{align*}

Then, we can establish our recursive relationship,
\begin{align*}
    a_{m+2}&=\frac{-3(1-m)}{(m+2)(m+1)}a_m\\
    &=\frac{-3}{m+2}a_m.\\
\end{align*}

We can use this relationship to verify $a_2$,
\[a_2=\frac{-3}{0+2}a_m=-\frac{3}{2}a_0.\]

We proceed to write several coefficients,
\begin{align*}
    a_2 &= \frac{-3}{2}a_0,\\
    a_3 &= \frac{-3}{3}a_1,\\
    a_4 &= \frac{-3}{4}\left( \frac{-3}{2} \right)a_0=\frac{3^2}{2^3}a_0, \\
    a_5 &= \frac{-3}{5}\left( \frac{-3}{3} \right)a_1=\frac{3^2}{3\cdot5}a_1,\\
    a_6 &= \frac{-3}{6}\left( \frac{3^2}{2^3} \right)a_0=\frac{-3^3}{2^4\cdot3}a_0,\\
    a_7 &= \frac{-3}{7}\left( \frac{2^2}{3\cdot5} \right)a_1=\frac{-2^2\cdot3}{3\cdot5\cdot7}a_1,\\
    a_8 &= \frac{-3}{8}\left( \frac{-3^3}{2^4\cdot3} \right)a_0=\frac{3^4}{2^7\cdot3}a_0,\\
    a_9 &= \frac{-3}{9}\left( \frac{-2^2\cdot3}{3\cdot5\cdot7} \right)a_0=\frac{-2^2\cdot3^3}{3^3\cdot5\cdot7}a_0,\\
    a_{10} &= \frac{-3}{10}\left( \frac{3^4}{2^7\cdot3} \right)=\frac{-3^5}{2^8\cdot3\cdot5}a_0,\\
    % a_{11} &= \frac{-3}{11}\left(  \right),\\
    % a_{12} &= \frac{-3}{12}\left(  \right)=\frac{3^6}{2^{10}\cdot3^2\cdot5}a_0,\\
    % a_{13} &= \frac{-3}{13}\left(  \right),\\
    % a_{14} &= \frac{-3}{14}.\left( \frac{3^6}{2^{10}\cdot3^2\cdot5} \right)a_0=\frac{-3^7}{2^{11}\cdot3^2\cdot5\cdot7}a_0,\\
\end{align*}

First, we will consider just the even coefficients $a_{2k}$, we see that they have a power of ${(-3)}^k$, in addition to the product of all even integers up to $2n$. This can be written in terms of factorials as $2^k k{!}$.

So, the rule for even coefficients is, \[a_{2k}=\frac{{(-3)}^k}{2^k k!}.\]

Next, the odd coefficients $a_{2k+1}$ also contain the $k^{\text{th}}$ multiple of $-3$, but their denominator is comprised of the product of all odd integers up to $2k+1$. This can be expressed using the double factorial,\footnote{The double factorial definition and expression with the regular factorial, \url{https://en.wikipedia.org/wiki/Double_factorial}.}.
\[1\cdot3\cdot5\cdot7\cdot\cdots\cdot(2k+1)=\frac{(2k+1)!}{2^k k!}.\]

So, the rule for odd coefficients is, \[a_{2k+1}=\frac{{(-3)}^k}{\frac{(2k+1)!}{2^k k!}}=\frac{{(-6)}^k k!}{(2k+1)!}.\]

\end{document}
