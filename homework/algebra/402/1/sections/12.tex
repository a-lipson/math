\documentclass[../hw1]{subfiles}
\begin{document}
\begin{problem}[12]
Solve the following for $x$:
\begin{enumerate}[label=\alph*)]
	\item $x^2+x=[0]$ in $\Z_5$
	\item $x^2+x=[0]$ in $\Z_6$
	\item Prove for $p$ prime,  the only solutions of $x^2+x=[0]$ in $\Z_p$ are $[0]$ and $[p-1]$.
\end{enumerate}
\end{problem}
\begin{proof}[Proof of (a)]
	By part (c), $x=[0],[4]$.
\end{proof}
\begin{proof}[Proof of b]
	By factoring, we have that \[
		x(x+1)\equiv 0 \pmod{6}
		.\]

	First, we have $x=[0]$.

	Then, we have the canonical representation factors of $6$ which are congruent to zero: {1,6}, {2,3}, and {3,4}.

	So,  $2(2+1)\equiv 0 \pmod{6} \implies x = [2]$ and $3(3+1)\equiv 0 \pmod{6} \implies x = [3]$

	However, 1 and 6 are not consecutive integers, so they do not satisfy the equation.

	Thus, $x = [0], [2], [3]$.
\end{proof}
\begin{proof}[Proof of c]
	By factoring, we have that \[
		x(x+1)\equiv0\pmod{p}
		.\]

	By Theorem 1.5, $x\equiv 0 \pmod{p}$ or $x+1\equiv 0 \pmod{p}$.

	So, $x=[0]$ or $x\equiv -1 \equiv p-1 \pmod{p}$.

	Thus, $x = [0], [p-1]$ are the solutions to the equation where $p$ is prime
\end{proof}
\end{document}
