\documentclass[../hw2]{subfiles}
\begin{document}
\begin{problem}[4]
Solve the following using exercise Hungerford 2.3.13.
\begin{enumerate}[label=\alph*)]
	\item $15x=9$ in  $\Z_{18}$.
	\item $25x=10$ in  $\Z_{65}$.
\end{enumerate}
\end{problem}
\begin{proof}[Proof of a]
	We have that $a=15,\, b=9,\, n=18$.
	Then, $(a,n)=(15,18)=3=d$; we check that $3|9 \implies d|b$.
	Next,
	\begin{align*}
		a=da' & \implies 15=3\cdot 5 \implies 5=a' \\
		b=db' & \implies 9=3\cdot 3 \implies 3=b'
		.\end{align*}

	By Theorem 1.8, $(15,18)=3$ affords a linear combination such that  $\exists u,v$ where \[
		15u+18v=3 \implies u=-1,\, v=1
		.\]

	Thus, by exercise (13), the solutions are [-3],[-3+18],[-3+36].

	Simplifying these classes into their canonical representations gives the single solution to the equation, [15].

	Therefore $x=15$ is a solution to  $15x=9$ in  $\Z_{18}$.
\end{proof}
\begin{proof}[Proof of b]
	We have $(25,65)=5$, so  $a=25,\, b=10,\, d=5,\, n=65$.

	Then, the linear combination $25u+65v=5 \implies u = -5,\,  v = 2$.

	Thus, by (13), the solutions are [-10],[-10+65],[-10+130],[-10+195],[-10+260].

	The canonical representation of these classes is [55].

	Therefore $x=55$ is a solution to the  $25x=10$ in  $\Z_{65}$.
\end{proof}
\end{document}
