\documentclass[../hw3]{subfiles}
\begin{document}
\begin{problem}
In $M(\C)$, let
$\bf{1}=\begin{pmatrix} 1 & 0 \\ 0 & 1 \end{pmatrix},\
	\mathbf{i}=\begin{pmatrix} i & 0 \\ 0 & -i \end{pmatrix},\
	\mathbf{j}=\begin{pmatrix} 0 & 1 \\ -1 & 0 \end{pmatrix},\
	\mathbf{k}=\begin{pmatrix} 0 & i \\ i & 0 \end{pmatrix}$.

Let the set $\mathbb{H}$ of real quaternions be \[
	a \mathbf{1}+b\mathbf{i}+c\mathbf{j}+d\mathbf{k} = \begin{pmatrix} a + bi & c + di \\ -c + di & a-bi \end{pmatrix},\quad a,b,c,d\in \R
	.\]
\begin{enumerate}[label=\alph*)]
	\item Prove
	      \begin{enumerate}[label=\roman*)]
		      \item $\mathbf{i}^2=\mathbf{j}^2=\mathbf{k}^2=-\mathbf{1}$
		      \item $\mathbf{i}\mathbf{j}=-\mathbf{j}\mathbf{i}=\mathbf{k}$
		      \item $\mathbf{j}\mathbf{k}=-\mathbf{k}\mathbf{j}=\mathbf{i}$
		      \item  $\mathbf{k}\mathbf{i}=-\mathbf{i}\mathbf{k}=\mathbf{j}$
	      \end{enumerate}
	\item  Show $\mathbb{H}$ is non-commutative and has an identity.
	\item  Show that $\mathbb{H}$ is a division ring.
	\item  Show that $x^2=-\mathbf{1}$ has infinitely many solutions in $\mathbb{H}$.
\end{enumerate}
\end{problem}
\begin{proof}[Proof of a]
	\begin{enumerate}[label=(\roman*)]
		\item Since $\mathbf{i}$ is a diagonal matrix, its square is the matrix with diagonal entries squared.
		      So, $\mathbf{i}^2=\begin{pmatrix} (i)^2 & 0 \\ 0 & (i)^2 \end{pmatrix} = \begin{pmatrix} -1&0\\0&-1 \end{pmatrix} = -\mathbf{1}$.

		      For $\mathbf{j}^2$ and $\mathbf{k}^2$, antidiagonal $2\times 2$ matrices square to a diagonal matrix with the entries as products of each original entry.
		      So, $\mathbf{j}^2=\begin{pmatrix} 1\cdot (-1)&0\\0&(-1)\cdot 1 \end{pmatrix} =-\mathbf{1}$ and $\mathbf{k}^2=\begin{pmatrix} i\cdot i&0\\0&i\cdot i \end{pmatrix} =-\mathbf{1}$.

		\item $\mathbf{i}\mathbf{j}=\begin{pmatrix} i & 0 \\ 0 & -i \end{pmatrix} \begin{pmatrix} 0&1\\-1&0 \end{pmatrix} = \begin{pmatrix} 0 & i \\ i & 0 \end{pmatrix}  = \mathbf{k}= -\begin{pmatrix} 0 & 1 \\ -1 & 0 \end{pmatrix}\begin{pmatrix} i & 0 \\ 0 & -i \end{pmatrix} = -\mathbf{j}\mathbf{i}$

		\item $\mathbf{j}\mathbf{k}=\begin{pmatrix} 0 & 1 \\ -1 & 0 \end{pmatrix}\begin{pmatrix} 0 & i \\ i & 0 \end{pmatrix}=\begin{pmatrix} i & 0 \\ 0 & -i \end{pmatrix}=\mathbf{i}=-\begin{pmatrix} 0 & i \\ i & 0 \end{pmatrix}\begin{pmatrix} 0 & 1 \\ -1 & 0 \end{pmatrix}=-\mathbf{k}\mathbf{j}$

		\item $\mathbf{k}\mathbf{i}=\begin{pmatrix} 0 & i \\ i & 0 \end{pmatrix}\begin{pmatrix} i & 0 \\ 0 & -i \end{pmatrix}=\begin{pmatrix} 0 & 1 \\ -1 & 0 \end{pmatrix}=\mathbf{j}=-\begin{pmatrix} i & 0 \\ 0 & -i \end{pmatrix}\begin{pmatrix} 0 & i \\ i & 0 \end{pmatrix}=-\mathbf{i}\mathbf{k}$
	\end{enumerate}
\end{proof}
\begin{proof}[Proof of b]
	Since matrix multiplication is non-commutative, then multiplication in $\mathbb{H}$, where $x \in \mathbb{H}$ is written in $2\times 2$ matrix form, is also non-commutative.

	Since we can express the $2\times 2$ identity matrix $I$ as  $1\mathbf{1} + 0\mathbf{i} + 0\mathbf{j} + 0\mathbf{k}$, then $1_{\mathbb{H}}=I$.
\end{proof}
\begin{proof}[Proof on c]
	Let $M=a\mathbf{1} + b\mathbf{i}+c\mathbf{j}+d\mathbf{k}$.
	Let $|M| =a^2 + b^2 + c^2 + d^2$.
	Let $N=\frac{1}{|M|}(a\mathbf{1} -b\mathbf{i}-c\mathbf{j}-d\mathbf{k})$.

	We will show that $N$ is the inverse of $M$, i.e.,  $MN=1$.
	Note that the product of any basis with the basis $\mathbf{1}$ remains the same.

	\begin{align*}
		MN & = ( a\mathbf{1} + b\mathbf{i}+c\mathbf{j}+d\mathbf{k} )\frac{1}{|M|}(a\mathbf{1} -b\mathbf{i}-c\mathbf{j}-d\mathbf{k}) \\
		   & = \frac{1}{|M|}(a^2 \mathbf{1}-ab\mathbf{i}-ac\mathbf{j}-ad\mathbf{k}                                                  \\
		   & + ab\mathbf{i}-b^2\mathbf{i}^2-bc\mathbf{i}\mathbf{j}-bd\mathbf{i}\mathbf{k}                                           \\
		   & +ac\mathbf{j}-bc\mathbf{j}\mathbf{i}-c^2\mathbf{j}^2-cd\mathbf{j}\mathbf{k}                                            \\
		   & + ad\mathbf{k}-bd\mathbf{k}\mathbf{i}-cd\mathbf{k}\mathbf{j}-d^2\mathbf{k}^2)                                          \\
		   & = \frac{1}{|M|}(a^2\mathbf{1}-b^2(-\mathbf{1})-c^2(-\mathbf{1})-d^2(-\mathbf{1})                                       \\
		   & = \frac{a^2+b^2+c^2+d^2}{a^2+b^2+c^2+d^2} = 1
		.\end{align*}
\end{proof}
\begin{proof}[Proof of d]
	Consider $x=0\mathbf{1} + b\mathbf{i}+c\mathbf{j}-d\mathbf{k}$ where $b^2 + c^2 + d^2=1$.

	Then,
	\begin{align*}
		x^2 & = (b\mathbf{i} + c\mathbf{j}-d\mathbf{k})(b\mathbf{i} + c\mathbf{j}-d\mathbf{k}) \\
		    & = b^2\mathbf{i}^2 + bc\mathbf{i}\mathbf{j}-bd\mathbf{i}\mathbf{k}                \\
		    & + bc\mathbf{j}\mathbf{i} +c^2\mathbf{j}^2-cd\mathbf{j}\mathbf{k}                 \\
		    & - bd\mathbf{k}\mathbf{i}-cd\mathbf{k}\mathbf{j}+d^2\mathbf{k}^2                  \\
		    & = b^2(-\mathbf{1}) + c^2(-\mathbf{1}) + d^2(-\mathbf{1})                         \\
		    & = (b^2 + c^2 +d^2)(-\mathbf{1})
		.\end{align*}

	But, with the vector length condition $b^2 + c^2 + d^2=1$, the above becomes just $-\mathbf{1}$.

	So, any $x$ satisfying the unit distance with zero $\mathbf{1}$ part will hold for $x^2=-\mathbf{1}$.

	Thus, $x^2=-\mathbf{1}$ has infinitely many solutions because $b^2 + c^2 + d^2=1$ has infinitely many solutions.
\end{proof}
\end{document}
