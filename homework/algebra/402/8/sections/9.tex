\documentclass[../hw8]{subfiles}
\begin{document}
\begin{problem}
\begin{enumerate}[label = \alph*)]
	\item Prove that the Gaussian Integers $\Z[i]$ are a subring of $\C$,
	      and prove that $M=\{a+bi \ \mid\ 3\mid a \land 3\mid b  \} $ is a maximal ideal in $\Z[i]$.
	\item Show that $\Z[i] / M$ is a field with nine elements.
\end{enumerate}
\end{problem}
\begin{proof}[Proof af a]
	Clearly, $\Z[i]\subset \C$.
	Then, there is a zero element, $0\in \Z[i]$.

	We have that $\Z[i]$ is closed under addition by the closure of $\Z$.

	We also have that \[
		(a+bi)(c+di)=(ac-bd)+(ad+bc)i
		,\] so $\Z[i]$ is closed under multiplication by the closure of $\Z$ under addition and multiplication.

	Therefore $\Z[i]$ is a subring of $\C$.

	Now, consider $z = a+bi \in \Z[i]$ where $z\not\in M$.

	$z \not\in M$ gives that either 3 does not divide $a$ or does not divide $b$.

	We will show that  $3$ does not divide  $N(z)=a^2 + b^2$.

	Suppose that $3\mid a^2 + b^2 = (a+bi)(a-bi)$.
	However, there is no such $a,b\in \Z$ with one not divisible by 3 which satisfy the above.

	So we must have that 3 does not divide the factor $a+bi$ either, which means that  $3\not\in M$.

	Since $3 \in \Z[i]$, then $M\subsetneq \Z[i]$.

	But, $1\in a+bi$ as $3$ does not divide  $1=a$.

	So any ideal containing such $a+bi$ and  $M$ must be  $\Z[i]$.

	Hence $M$ is maximal.
\end{proof}
\begin{proof}[Proof  of b]
	Since $M$ is maximal, then  $\Z[i]/M$ is a field.

	Consider the cosets $(a+bi)+M\in \Z[i] / M$.
	There are three choice for  each $a$ and $b$, \[
		0,\, 1,\, 2,\, i,\, 2i,\, 1+i,\, 2+i,\, 1+2i, 2+2i
		,\] these canonical representations form the congruence classes of the field $\Z[i]/M$
\end{proof}
\end{document}
