\documentclass[../hw2]{subfiles}
\begin{document}
\begin{problem}[4]
Let $x_1=\sqrt{2},x_{n+1}=\sqrt{2+x_{n}}$

\begin{enumerate}[label=\alph*]
	\item Prove $\forall n, \ x_n<2$ and $x_{n+1}>x_{n}$.
	\item Prove ${(x_{n})}_{n=1}^{\infty}$ converges, and find the limit.
\end{enumerate}
\end{problem}
\begin{proof}[Proof of a]
	We will prove the statement by induction on $n$. Let  $n=1$, then  $x_1=\sqrt{2} < 2 $. Next, \[
		x_2=\sqrt{2+x_1}=\sqrt{2+\sqrt{2} }>\sqrt{2}=x_1
		,\] which implies that $x_2>x_1$.

	Now, assume that the statement holds for $n=k$,\[
		x_k<2\ \text{and}\ x_{k+1}>x_k
		.\]

	Then, \[
		x_{k+1}=\sqrt{2+x_k}<\sqrt{2+2}=2
		.\] So $x_{k+1}<2$.

	Also, \[
		2x_{k+1}>{(x_{k+1})}^2=2+x_k>2x_k
		.\] So $x_{k+1}>x_k$.

	Since the base case $n=1$ holds, and  $n=k+1$ holds where  $n=k$ holds, then the statement holds for all  $n$.
\end{proof}
\begin{proof}[Proof of b]
	Since $\forall n,\ x_n<2$, then $(x_n)$ is bounded above by 2. Since  $\forall n,\ x_{n+1}>x_n$, $(x_n)$ is monotonically increasing. Thus,  $(x_{n})$ converges by the MBST to its upper boundary.
\end{proof}
\begin{proposition}
	$(x_n)$ converges to 2, $\sup(x_n)=2$.
\end{proposition}
\begin{proof}[Proof of Proposition]
	Suppose, for a contradiction, $\exists\, b < 2$ such that $\sup(x_n)=b$.

	Then, $\forall \epsilon>0,\ \exists x_n,\ b-\epsilon<x_n<b<2$. But, since $b$ is  $\epsilon$ close to 2 for any small  $\epsilon$, then  $b=2$, contradicting the assumption that there was a supremum $b$ smaller than 2.
\end{proof}
\end{document}
