\documentclass[../9extra]{subfiles}
\begin{document}
\begin{problem}[2]
Give an example of a shrinking map that is not a contraction map.
\end{problem}
\begin{proof}[Proof of Problem 2]
	Since a contraction map requires, for some fixed $\alpha \in (0,1)$, that $\forall x,y \in K,\, x \neq y$, \[
		|f(x)-f(y)| < \alpha|x-y|
		,\]
	then we wish to find a map such that this relationship will not hold for any fixed choice of $\alpha$.

	$f:[0,1]\longrightarrow \R$ defined as $f(x)=x - \frac{x^2}{2}$ is a shrinking map that is not a contraction map.

	As $x$ approaches zero, $f'(x)=1-x$ will get arbitrarily close to 1, so no fixed $\alpha$ will work.
\end{proof}
\end{document}
