\documentclass[../hw1]{subfiles}
\begin{document}
\begin{problem}[5]
Let $f(x)=1$ on $[n,n+2^{-n}]$ for $n \in \Z_{>0}$ and $f(x)=0$ elsewhere.
\begin{enumerate}[label=\alph*)]
	\item Show $\int_{0}^{\infty} f(x) \,dx$ converges to one, yet $f(x)\not{\to}0$ as $x\to \infty$.

	      The integral of $f$ can actually be expressed using a sum, since the area under the graph of $f$ consists of rectangles with height one and with  $2^{-n}$, which is the length of each interval where $f(x)=1$.

	      Thus, by the explicit formula for the geometric series,
	      $\int_{0}^{\infty} f(x) \,dx = \sum_{1}^{\infty} 2^{-n} = \sum_{0}^{\infty} {(\frac{1}{2})}^n - 1 = \frac{1}{1-\frac{1}{2}} - 1 = 1$.

	\item Modify $f$ to a function  $g$ such that  $\int_{0}^{\infty} g(x) \,dx$ converges yet $g$ is not bounded as  $x\to \infty$.

	      If we desire $g$ to be unbounded, then we will need to shrink the intervals where  $g\neq 0$ so that the increasing value of $g$ has a proportionally decreasing contribution to the final integral.

	      Let  $g(x)=\begin{cases}
			      x, & \forall n \in \Z_{>0},\, x \in [n,n+\frac{1}{2^n n}], \\
			      0, \text{otherwise}.
		      \end{cases}$

	      So, $\int_{0}^{\infty} g(x) \,dx \approx \sum_{1}^{\infty} 2^{-n}\cdot \frac{n}{n}$, which converges as seen above.

	      Furthermore, we can create an inequality by considering that each $n$th interval will have a value of  $g$ less than  $n+1$.
	      This gives  \[
		      \int_{0}^{\infty} g(x) \,dx < \sum_{1}^{\infty} 2^{-n}\left( \frac{n+1}{n} \right) < \infty
		      .\]

	      So the integral of $g$ converges while $g$ grows without bound.
\end{enumerate}
\end{problem}
\end{document}
