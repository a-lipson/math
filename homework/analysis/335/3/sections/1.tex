\documentclass[../hw3]{subfiles}
\begin{document}
\begin{problem}
Compute $\lim_{k \to \infty} f_k$ on the given interval $I$
Is $f_k$ uniformly convergent on  $I$?
If not, is it uniformly convergent on a smaller interval?
\begin{enumerate}[label=\roman*)]
	\item $f_k(x)=x^{\frac{1}{k}},\, x \in [0,1]$.
	\item $f_k(x)=kxe^{-kx},\, x \in [0,infty)$.
	\item $f_k(x)=\frac{x}{k}e^{-\frac{x}{k}},\, x \in [0,\infty)$.
	\item $f_k(x)=\frac{x^k}{1+x^{2k}},\, \forall x \in [0,\infty)$.
\end{enumerate}
\end{problem}
\begin{proof}[Proof of i]
	We see that at $x=0$, then  $\forall k,\, x^{\frac{1}{k}}=0$.
	For $x \in (0,1]$, \[
		\lim_{k \to \infty} f_k(x)=\lim_{k \to \infty} e^{\log{f_k(x)}}=\lim_{k \to \infty} e^{\log{x^{1 / k}} } =\lim_{k \to \infty} e^{\frac{1}{k}\log{x} }=e^0=1
		.\]

	So $f_k\to \begin{cases}
			0, & x=0, \\ 1, & x \in (0,1].
		\end{cases}$

	We will now check if $f_k$ converges uniformly.
	Already, since $f_k$ converges to a discontinuous function, then $f_k$ will not converge uniformly.

	We see that the convergence of $x^{\frac{1}{k}}$ to 1 will have $k$ depend on  $x$.
	If we fix $k=$ so that $\sqrt[k]{x} = 1$, then we can consider $\frac{x}{2}$ where ${\left( \frac{x}{2} \right)}^{1 / k} = {\left( \frac{1}{2} \right) }^{1 / k} < 1$.
	Thus, $k$ depends on $x$ so $\{ f_k \}$ is not uniformly convergent.

	Now, consider $\forall \epsilon>0,\,  x \in [\epsilon,1]$,  \[
		\sup_{x \in [\epsilon,1]}|f_k(x)-f(x)| = \sup_{x \in [\epsilon,1]}|x^{\frac{1}{k}}-1|
		.\]
	We can fix $k$ such that $\epsilon^{\frac{1}{k}}=1$, and this same $k$ will work for all  $\epsilon<x\le 1$ as well.
\end{proof}
\begin{proof}[Proof of ii]
	First, we see that \[
		\lim_{k \to \infty} f_k(x)=\lim_{k \to \infty} e^{\log{\left( kxe^{-kx} \right) } }=\lim_{k \to \infty} e^{-kx\log{(kxe)} } = 0
		.\]

	For convergence, we will find the maximum of $f_k$.
	\[
		f'_k(x)=ke^{-kx}(1-kx)=0\implies x=\frac{1}{k}
		.\]
	So, the max of $f_k$ is $f_k\left( \frac{1}{k} \right) = e^{-1}$.

	Note that, as $k\to \infty,\, \frac{1}{k}\to 0$.

	So, we have that, as $k\to \infty$, \[
		\sup_{x \in [\epsilon,\infty)} \left|kxe^{-kx}\right| \to 0
	\] where $\epsilon>\frac{1}{k}\to 0$ as we have a fixed $k$.
\end{proof}
\begin{proof}[Proof of iii]
	We see that, as $k\to \infty,\, f_k\to 0$.
	Then, for the max of $f_k$,  \[
		f'_k(x)=\frac{1}{k}e^{-\frac{x}{k}}\left( 1-\frac{x}{k} \right) =0 \implies x = k
		.\]

	Since $f_k(0)=0$, $f_k\to 0$, and $f_k>0$ for all $x$ in the given domain, then the critical point $x=k$ is the supremum of $\frac{x}{k}e^{-\frac{x}{k}}$ will occur at $x=k$.

	But, as $k\to \infty$, \[
		\sup_{x \in [0,\infty)} \left|\frac{x}{k}e^{-\frac{x}{k}}\right| \to 0
		.\]

	So, $f_k\rightrightarrows 0$.
\end{proof}
\begin{proof}[Proof of iv]
	Note that $f_k(0)=0$.
	Then, as $k\to \infty$, \[
		\begin{array}{r l}
			x<1   & \implies f_k(x) \approx x^k \to 0,    \\
			x = 1 & \implies f_k(1) = \frac{1}{2},        \\
			x > 1 & \implies f_k(x) \approx x^{-k} \to 0. \\
		\end{array}
	\]
	So, we see that $f_k$ will converge uniformly on $x \in \left[0,\frac{1}{2}\right)$ and $x \in \left(\frac{1}{2}, \infty\right)$.
\end{proof}
\end{document}
