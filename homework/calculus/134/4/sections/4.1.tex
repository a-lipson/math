\documentclass[../hw4.tex]{subfiles}

\begin{document}

\subsection*{29}
A number $c$ is called a fixed point of $f$ if $f(c) = c$. 
Prove the proposition.
%HINT: Form $g(x) = f (x) − x$.

\begin{proposition}
    If $f$ is differentiable on an interval $I$ and $f'(x) < 1$ for all $x \in I$, 
    then $f$ has at most one fixed point in $I$. 
\end{proposition}

\begin{proof}
    For a contradiction, assume that $f$ has two fixed points on $I$.

    Define $g(x) = f(x) - x$ so that when $x$ is a fixed point of $f$, $g = 0$.

    Since $f$ is differentiable, on $I$, it is also continuous on $I$. So, $g$ as the sum of a two continuous functions is also continuous.

    Similarly, since $f$ and $-x$ are both differentiable, $g$ is differentiable.

    So, \[g'(x) = f'(x) - 1.\]

    But, $f'(x) < 1$ was given. So, \[g'(x) < 0.\]

    Then, by the assumption, if $f$ has two fixed points on $I$, then $g$ has two zeros on $I$. 

    Let $a,b \in I$ be defined such that $g(a)=g(b)=0$.

    Then, there exists $c$ in $I$ between $a$ and $b$ such that $g'(c) = 0$ by Rolle's Theorem.

    But, $g'(x) < 0$ and $g'(c) = 0$ is a contradiction.

    So the proposition is true.
\end{proof}


\subsection*{34}
Show that the proposition (2) holds.
%Show that the equation $x^n + ax + b = 0$, $n$ an odd positive integer, has at most three distinct real roots.

% \begin{proposition}[1]\label{proposition:one}
%     If $m = 2n : n \in \mathbb{Z}$ and $y^m = x$, then $y = \pm \sqrt[m]{x}$.
% \end{proposition}

% \begin{proof}[Proof of (1).]
%     We will prove the proposition using induction.

%     For the base case, when $m=2$,
%     \begin{align*}
%         y^2 &= x \\
%         \sqrt{y^2} &= \sqrt{x} \\
%         |y| &= \sqrt{x} \\
%         y &= \pm \sqrt{x} \\
%     \end{align*}

%     Assume that the proposition holds for an even $k$ and $m=k$ by the inductive hypothesis.

%     For the inductive step, we let $m=k+2$,
%     \begin{align*}
%         y^{k+2} &= x \\
%         {\left( y^{k+2} \right)}^{\frac{1}{k+2}} &= x^{\frac{1}{k+2}} \\
%         %\sqrt[k+2]{y^{k+2}} &= \sqrt[k+2]{x} \\
%         {\left( y^{k}y^{2} \right)}^{\frac{1}{k+2}} &= x^{\frac{1}{k+2}} \\
%         {\left(y^k\right)}^{\frac{1}{k+2}} {\left(y^2\right)}^{\frac{1}{k+2}} &= x^{\frac{1}{k+2}} \\
%         y^{\frac{k}{k+2}} y^{\frac{2}{k+2}} &= x^{\frac{1}{k+2}} \\
%         y^{\frac{k+2-2}{k+2}} y^{\frac{2}{k+2}} &= x^{\frac{1}{k+2}} \\
%         y^{1-\frac{2}{k+2}} y^{\frac{2}{k+2}} &= x^{\frac{1}{k+2}} \\
%     \end{align*}
% \end{proof}

\begin{proposition}%[2]
    If $f(x) = x^m+ax+b$, and $m = 2n+1 : n \in \mathbb{Z}^+$, then $f$ has at most three distinct real roots.
\end{proposition}

\begin{proposition}[1]
    If $g$ is a differentiable function and $g'$ has no more than $k$ roots, then $g$ has no more than $k+1$ roots.
\end{proposition}

\begin{proof}[Proof of (1).]
    For a contradiction, assume that $g$ has $k+2$ roots.

    Then, by Rolle's Theorem, $g'$ has $k+1$ roots, contradiction the assumption that $g'$ had $k$ roots.
\end{proof}

\begin{proof}%[Proof of (2).]
    First, we will differentiate $f$ twice.
    \begin{align*}
        f(x) &= x^m+ax+b \\
        f'(x) &= mx^{m-1}+a \\
        f''(x) &= m(m-1)x^{m-2} \\
    \end{align*}

    If $m=1$, then $f$ is a linear function $(1+a)x+b$, which only has one root, so the proposition holds. So, we continue with all positive odd integers such that $m>1$.

    We see that $f''$ can only have one root at $x=0$.

    Therefore, by proposition (1), $f'$ has no more than two roots.

    Similarly, since $f'$ has no more than two roots, $f$ has no more than three roots, again by proposition (1).
 
    Therefore $f$ has at most three distinct roots.
    % For a contradiction, assume that $f$ has four roots. 

    % Then, by Rolle's Theorem, there are three unique values $x_1,x_2,x_3$ such that $f'(x_1)=f'(x_2)=f'(x_3)=0$.

    % Next, we differentiate $f$, \[f'(x) = mx^{m-1}+a.\]

    % Then, we find the roots of $f'$,
    % \begin{align*}
    %     mx^{m-1}+a &= 0 \\
    %     x^{m-1} &= \frac{-a}{m}
    % \end{align*}

    % Since $m$ is odd, then $m-1$ is even. So, by proposition (1),
    % \[x = \pm {\left( \frac{-a}{m} \right)}^{\frac{1}{m-1}}.\]

    % Therefore, $f'(x)$ has at most two real $x$\footnote{If $a$ = 0, then $f'$ has one root $x=0$.} such that $f'(x) = 0$, which contradicts the fact that $f'$ has three values of $x$ such that $f'(x) = 0$.
    
    % Therefore, by contradiction, the proposition holds.

    % % QUESTION can we use Rolle backwards, e.g. one f'=0 implies f has two roots?
\end{proof}



\end{document}