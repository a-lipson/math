\documentclass[../hw6.tex]{subfiles}

\begin{document}

\subsection*{16}
Determine whether the assertion is true or false on an arbitrary interval $[a,b]$ on which $f$ and $g$ are continuous.

We choose to denote the average of a function $f$ as $\overline{f}$ instead of $f_{avg}$ as is used in the textbook.


\begin{definition}
    By 5.9.1, the average of a function $h$ is defined as, 
    \[\int_{a}^{b} h(x) dx = \overline{h}(b-a)\]
    where $\overline{h}=h(c)$ for some $c \in [a,b]$.
    
    We see that, \[\frac{1}{b-a} \int_{a}^{b} h(x) dx.\]    
\end{definition}


\begin{enumerate}[label= (\alph*)]
    \item $\overline{f+g}=\overline{f}+\overline{g}$
    
    Using the definition with $h(x)=f(x)+g(x)$ and 5.4.4, we see that,
    \[ \frac{1}{b-a}\int_{a}^{b} f(x)+g(x) dx = \frac{1}{b-a} \int_{a}^{b} f(x) dx + \frac{1}{b-a} \int_{a}^{b} g(x) dx,\]
    which is $\overline{f}+\overline{g}$ by the definition.


    \item $\overline{\alpha f}=\alpha \overline{f}$
    
    Using the definition and $h(x)=\alpha f(x)$ and 5.4.3, we get that
    \[\frac{1}{b-a} \int_{a}^{b} \alpha f(x) dx = \alpha \frac{1}{b-a} \int_{a}^{b} f(x) dx,\]
    which is $\alpha \overline{f}$ by the definition.

    \item $\overline{fg}=\overline{f}\cdot\overline{g}$
    
    With $h(x)=f(x)g(x)$, by the definition, \[\overline{fg} = \frac{1}{b-a}\int_{a}^{b} f(x)g(x) dx.\]

    We will show that the statement does not always hold by providing a counterexample.

    Let $f(x)=g(x)=x$. 
    
    Let interval of integration $[a,b]$ be $[0,1]$.

    So, $\frac{1}{b-a} = 1$.

    Then, \[\overline{fg} = 1\cdot\int_{0}^{1} x\cdot x \, dx = \frac{x^3}{3} \bigg\vert_{0}^{1} = \frac{1}{3}.\]
    
    But, \[\overline{f}\cdot\overline{g} = {\left[ \int_{0}^{1} x\,dx \right]}^2 = {\left[ \frac{x^2}{2} \bigg\vert_{0}^1 \right]}^2 = {\left[ \frac{1}{2} \right]}^2 = \frac{1}{4}.\]

    Since, $\frac{1}{3} \neq \frac{1}{4}$, then, for some continuous functions $f$ and $g$ and a closed and bounded interval, $\overline{fg}\neq\overline{f}\cdot\overline{g}$.

    So, the assertion is not always true.

    \item $\overline{fg}=\overline{f}/\overline{g}$
    
    Again, we will show that the statement does not always hold by providing a counterexample.

    Using the same $f$ and $g$ as in the previous statement, we have that $\overline{fg}=\frac{1}{3}$.

    But, with $\overline{f}=\overline{g}=\frac{1}{2}$, \[\overline{f}/\overline{g} = \frac{\frac{1}{2}}{\frac{1}{2}} = 1.\]

    Since $\frac{1}{3} \neq 1$, then $\overline{fg}\neq\overline{f}/\overline{g}$ for some continuous functions $f$ and $g$ on a closed and bounded interval. 

    So, the assertion does not always hold for all continuous functions $f$ and $g$ on an arbitrary closed and bounded interval.

\end{enumerate}



\end{document}