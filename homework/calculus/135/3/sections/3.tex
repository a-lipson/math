\documentclass[../hw3]{subfiles}

\begin{document}

\subsection*{12.5.42}
\textit{Alternating Series Test without decreasing condition}.
Find an example where $\lim\limits_{n\to\infty}a_n=0$, $a_n\geq0$ but the alternating series $\sum {(-1)}^k a_k$ diverges.

If we alternate between a divergent series of positive terms with a convergent series of positive terms, we will wind up with a diverging alternating series whose terms are all positive. 

For simplicity, we will choose the constant zero series as our convergent series, and the harmonic series as our divergent series. 

We define our new series $a_k$ as the combination of the two aforementioned series, where the odd terms are from the zero series and the even terms come from the harmonic series.
\[a_k=1-0+\frac{1}{2}-0+\frac{1}{3}-0+\cdots.\]

We can more clearly represent this series by its odd and even components, as we did when constructing the series in the first place, being careful of division by zero with the first-indexed term,
\[a_0=1, \quad a_{2k}=\frac{2}{k}, \quad a_{2k+1}=0, \quad \forall k > 0.\]

If we consider the convergence of sequence, $a_k$, we note that all the nonzero terms have a corresponding term in the sequence $\frac{1}{k}$, which converges to zero as $k\to\infty$. Additionally, all the these terms are positive. This satisfies the conditions of $a_k$.

Then, for $\sum {(-1)}^k a_k$, any odd $k$ will provide a negative coefficient, yet the value of $a_k$ will be zero. With even $k$, the positive values of the harmonic sequence will be added to the series with positive coefficients, which will cause the alternating series with $a_k$ to diverge, just as the harmonic series does.\footnote{But, perhaps only half as fast? Can we speak about the speed of divergence?} 

If we desire a more explicit formula, we can consider the square of the sine or cosine functions, which have ranges between zero and one. 

Take $f(k) = \cos^2{\left( \frac{\pi}{2}k \right)}$.

We see that, \[f(k)=\begin{cases}
    1,& k \text{ even}, \\
    0,& k \text{ odd}.
\end{cases}\]

We can then combine this function with the ``harmonic sequence''\footnote{I will call the sequence of the harmonic series' terms the harmonic sequence for the remainder of this excessively long example.} $1/k$ to produce a new sequence with only odd or even terms.

We will consider the sequence,
\begin{align*}
    b_k=\frac{\cos^2{\left( \frac{\pi}{2}k \right)}}{k}.
\end{align*}

This sequence contains only positive terms and will tend toward zero as $k\to\infty$ by the comparison test (12.3.6) with the harmonic sequence $1/k$. As such, it is a valid candidate for this example.

Then, we note, for all odd $k$, where the coefficient will be negative, the value of $b_k$ is zero. This is the harmonic series with only even denominators Thus, the sequence of partial sums of the series $\sum {(-1)}^k b_k$ will always grow. 

In fact, the harmonic series with only even denominators, $\sum \frac{1}{2k}$, is clearly related to the harmonic series itself, in that the above is, \[\frac{1}{2} \sum \frac{1}{k}.\] Since the harmonic series diverges, so too does any constant multiple of the series (12.3.5). 

A similar procedure can be performed for all even $k$, noting that we could consider instead $\sin^2{\left( \frac{\pi}{2} k \right)}$.

\end{document}