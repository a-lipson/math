\documentclass[../hw7]{subfiles}
\begin{document}
\begin{problem}
\item An \emph{$n$-gonal pyramid} is a polyhedron formed by connecting each vertex of an $n$-sided polygon with one additional vertex.
\begin{enumerate}
	\item What is the dual polyhedron of an $n$-gonal pyramid?
	\item What is the chromatic number of an $n$-gonal pyramid?
\end{enumerate}
\end{problem}
\begin{proposition}
	An $n$-gonal pyramid is its own dual.
\end{proposition}
\begin{proof}[Proof of Proposition and (a)]
	Let $N$ be the $n$-gonal pyramid and  $N^*$ be its dual.

	Then, for  $N$, $V=n+1$ for the vertices in the base  $n$-gon and the apex.

	$F=n+1$ for the  $n$-gon base and the $n$ triangular faces.

	$E=2n$ for the  $n$ edges in the base $n$-gon and the  $n$  edges connecting the base vertices to the apex.

	Next, for $N^*$, there is a vertex at each face of  $N$, so $V^*=F=n+1$.

	Similarly, there is a face on each vertex of  $N$, so  $F^*=V=n+1$.

	Each edge in $N$ corresponds to an edge in the dual $N^*$, so  $E^*=N=2n$.

	In fact,  $N^*$  has the same structure as $N$.

	The vertex in the face of the base $n$-gon in  $N$ is the apex in  $N^*$.

	The  $n$ vertices on triangular faces of $N$ form the vertices of the $n$-gon base in  $N^*$.

	Thus,  $N\cong N^* $ implies that an $n$-gonal pyramid is its own dual.
\end{proof}
\begin{proof}[Proof of (b)]
	By Problem 3, since the maximum degree in $N$ belongs to the apex vertex with  degree $n$, then $\chi(N)=n+1$.
\end{proof}
\end{document}
