\documentclass[../hw7]{subfiles}
\begin{document}
\begin{problem}
\item Let $G_n$ be the graph with vertex set $\{0,1\}^n$ and an edge between $x$, $y \in \{0,1\}^n$ if $x$ and $y$ differ at exactly 2 positions.
\begin{enumerate}
	\item Prove that $\chi(G_3) = 4$ and $\chi(G_4) = 4$.
	\item Prove that $\chi(G_n) \ge n$ for all positive integers $n$.
\end{enumerate}
\end{problem}
We have that the vertices of $G_n$ are $n$-bit binary strings.

Define the parity of a binary string to be the number of ones modulo 2;
i.e., strings with an even parity have an even number of ones,
and those with an odd parity have an odd number of ones.

Note that, if two binary strings differ by exactly one position, then the must have opposite parities.
If two binary strings differ by exactly two positions, then they must have the same parity.

\begin{proof}[Proof of (a)]
	$G_3$ contains two connected components, each of which are $K_4$.
	Since $K_4$ is 4-colorable, then $\chi(G_3)=4$ as well.

	% We will partition the vertices of $G_4$ into four components such that each components has no edges between its vertices.
	For $G_4$, we will split the 16 vertices into two sets, $S_0$ and $S_1$, by whether the first bit in the vertex binary string is 0 or 1.
	Each set contains an isomorphic copy of $G_3$ where each vertex binary string is prefixed by either 0 or 1.

	Since we already know $G_3$ to be 4-colorable, then $G_4$ must be at least 4 colorable as well.
	Color the subgraph formed by $S_0$ with the same coloring $c_4$ that works for $G_3$.

	Note that this coloring ensures that adjacent vertices with the same parity get different colors.

	Now, we must connect the vertices between the subsets $S_0$ and $S_1$ of $G_4$.

	Since the first vertex of each subset already differs by one position,
	then we must place edges between $v_0 \in S_0$ and $v_1 \in  S_1$ where the last 3-bits of the binary strings differ by exactly one position.

	So, we will connect vertices whose last 3-bits differ in parity.
	This connects vertices in $S_0$ with vertices in $S_1$ with opposing parity.

	Hence, each vertex of $S_0$ will be connected to a corresponding vertex in $S_1$ which it was not connected to in $S_0$, since vertices in $S_0$ are connected to those of the same parity in $S_0$.

	Thus, we can take the coloring $c_4$ from  $S_0$ and swap the colors used for even and odd parity vertices (we have two choices for each parity because there are four colors).
	This will ensure that the vertices connected between $S_0$ and $S_1$ have different colors.

	Hence we still have a 4-coloring in $G_4$ and $\chi(G_4)=4$.
\end{proof}
\begin{proof}[Proof of (b)]
	We will induct on $n$.
	For the base case, we have that $\chi(G_1)=1$ as there are two disconnected vertices.

	We have that $G_2$ is a 4-cycle, which is bipartite and hence 2-colorable; i.e., $\chi(G_2)=2$.

	We also have that $\chi(G_3)=\chi(G_4)=4$ by part (a).

	Now, assume that the statement holds for $n\ge 5$.

	Consider a partition $\{S_0,S_1\} $ of the vertices of $G_{n+1}$ based on the first bit of the vertex binary string as before in part (a).

	Now, $G_{n+1}$ contains two subgraphs, each of which are $n$-colorable by the inductive hypothesis.

	Then, vertices are connected between the subgraphs if the tail $n$-bit binary strings differ by exactly one bit since these vertices already differ by the first bit.

	So, the vertices in $S_0$ must be connected to vertices of the opposite tail parity in $S_1$, and they were not connected to the corresponding vertex in $S_0$.

	Suppose that $G_{n+1}$ was only $n$-colorable, with $n$ colors for $S_{0}$ and $S_1$ each, possible permuting colors to avoid conflicts.

	Suppose we assign a coloring $c_n$ to $S_0$ such that that all even-parity vertices use a subset of the $n$ colors and all odd-parity vertices use a different subset of the $n$ colors.

	However, when we attempt to copy this coloring $c_n$ to $S_1$, we see that an even-parity vertex in $S_0$ must now be adjacent to an odd-parity vertex in $S_1$.

	Since these two vertices were not adjacent in $S_0$, they may have been given the same color originally.

	But now, they are forced to be adjacent in $G_{n+1}$, so they must have different colors.

	Since this occurs for all pairs, we need an extra color.
	% However, since the connections between $S_0$ and $S_1$ force opposite-parity vertices to have different colors, then this introduces at least one more distinct color.

	Therefore, $G_{n+1}$ must require at least $n+1$ colors.

	Thus, $\chi(G_{n+1})\ge n+1$.
\end{proof}
\end{document}
