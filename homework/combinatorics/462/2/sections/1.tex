\documentclass[../hw2]{subfiles}
\begin{document}
\begin{problem}
Let $f:\N\to \R$ such that \[
	f(n)=f(n-1)+2f(n-2)+2^n
\] for $n\ge 2$, $f(0)=1$, and $f(1)=2$.
\begin{enumerate}[label=(\alph*)]
	\item Find a closed form expression for the generating function $F(x)$ of $f(n)$.
	\item Prove that $f(n)\sim cn 2^n$ for some constant $c$.
	      % Hint: use partial fraction decomposition without solving for numerator coefficients.
\end{enumerate}
\end{problem}
\begin{proof}[Proof of (a)]
	We will form a generating function using the recurrence relation,
	\begin{align*}
		\sum_{0}^{\infty} f(n+2)x^n                 & = \sum_{0}^{\infty} f(n+1)x^n + 2 \sum_{0}^{\infty} f(n)x^n + \sum_{0}^{\infty} 2^{n+2}x^n \\
		\frac{1}{x^2}\left( F(x)-xf(1)-f(0) \right) & = \frac{1}{x}(F(x)-f(0))+2F(x)+\frac{4}{1-2x}                                              \\
		F(x)-2x-1                                   & = xF(x)-x+2x^2F(x)+\frac{4x^2}{1-2x}                                                       \\
		F(x)(1-x-2x^2)                              & =1+x+\frac{4x^2}{1-2x}                                                                     \\
		F(x)(1+x)(1-2x)                             & = \frac{(1-2x)(1+x)+4x^2}{1-2x}                                                            \\
		F(x)                                        & = \frac{1+x-2x+2x^2}{(1-2x)^2(1+x)}
		,\end{align*} which is our closed form for $F(x)$.
\end{proof}
\begin{proof}[Proof of (b)]
	Since the degree of the numerator of the closed form of $F$ is less than that of the denominator, then the quotient remainder of $F$ can be expressed using partial fractions with constant numerators.

	So, we have \[
		\frac{c_1}{1-2x}+\frac{c_2}{(1-2x)^2}+\frac{c_3}{1+x}
		.\]
	Note that \[
		\frac{1}{(1-y)^2}=\frac{d}{dx}\left( \frac{1}{1-y} \right) = \frac{d}{dx}\sum_{n\ge 0} x^n = \sum_{n\ge 0} (n+1)x^n
		.\]
	Therefore, using the geometric power series expansion, we have
	\begin{align*}
		F(x) & = c_1 \sum_{n\ge 0} (2x)^n + c_2 \sum_{n\ge 0} (n+1)(2x)^n + c_3 \sum_{n\ge 0} c_3 \sum_{n\ge 0} (-x)^n \\
		     & = \sum_{n\ge 0} \left( c_1 2^n + c_2(n+1) 2^n + c_3 (-1)^n \right) x^n                                  \\
		     & = \sum_{n\ge 0} (c_2n + (c_1+c_2))2^n + c_3(-1)^n x^n
		.\end{align*}
	Thus,  \[
		f(n)=(c_2n + (c_1+c_2))2^n + c_3(-1)^n \implies f(n) \sim cn 2^n
		,\] as desired.
\end{proof}
\end{document}
