\documentclass[../astro_4]{subfiles}
\begin{document}
\begin{problem}
Which gases from Table 3 could Ceres retain?
\end{problem}
We will repeat the procedure used in problem 6.

First, we will compute the maximum speed that gas molecules can have in order to remain in Ceres' atmosphere for geologic time.
\[
	\frac{1}{6}510 = 85\text{ m/s}
	.\]

We will use the same temperature distance model to determine the temperature of Ceres at a distance of 2.8 AU, this is about 238 K.

Next, we will fill in the table,
\[
	\begin{array}{c | c}
		\text{gas}         & \text{average velocity (m/s)}        \\
		\hline
		\text{H}_2         & 157\sqrt{\frac{238}{2}} \approx 1713 \\
		\text{CH}_4        & 157\sqrt{\frac{238}{16}} \approx 606 \\
		\text{NH}_3        & 157\sqrt{\frac{238}{17}} \approx 587 \\
		\text{H}_2\text{0} & 157\sqrt{\frac{238}{18}} \approx 571 \\
		\text{N}_2         & 157\sqrt{\frac{238}{28}} \approx 458 \\
		\text{CO}_2        & 157\sqrt{\frac{238}{44}} \approx 365 \\
	\end{array}
\]
Since all of these gasses would have an average velocity greater than Ceres' atmospheric capture velocity of 85 m/s,
then Ceres could retain any of the gasses from Table 3.
\end{document}
