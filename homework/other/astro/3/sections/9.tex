\documentclass[../astro_4]{subfiles}
\begin{document}
\begin{problem}
Jupiter currently orbits the Sun at 5 AU.
Could Jupiter retain an atmosphere of hydrogen molecules if you could magically move Jupiter to 0.5 AU from the Sun?
Why or why not?
Explain your answer.
\end{problem}

Currently, Jupiter can retain hydrogen since it is very massive and has a strong gravity; but, at the same time, its large distance from the Sun also cools the hydrogen gas molecules, giving them less kinetic energy and velocity.

So, following the procedure of problems 6 and 7, we will first compute the maximum velocity for a gas retained through geologic time, \[
	\frac{1}{6}59500\approx 9917\text{ m/s}
	.\]
Now, we will determine the temperature that Jupiter at 0.5 AU from the Sun would have.
This is given directly by the table as 566 K.

So, we will build out the same table as before,
\[
	\begin{array}{c | c}
		\text{gas} & \text{average velocity (m/s)}        \\
		\hline
		\text{H}_2 & 157\sqrt{\frac{566}{2}} \approx 2641 \\
		\vdots     & \vdots                               \\
	\end{array}
\]

Since even hydrogen, the lightest gas, would have insufficient velocity to escape Jupiter, even at the increased temperature at 0.5 AU from the Sun, then all the heavier gasses would be retained as well.

Thus, Jupiter would retain at atmosphere with hydrogen molecules at a distance of 0.5 AU.

In order to force Jupiter to be too hot to retain its hydrogen, we can consider the following,
\[
	9917 \le 157\sqrt{\frac{T}{2}} \implies {\left(\frac{9917\sqrt{2}}{157}\right) }^2= T
	.\]

Recalling the regression for temperature $T$ K in terms of distance $r$ AU, $T=\frac{392.41373}{\sqrt{r}}+3.83379$, we can combine with the above equation to solve for the distance $r$.
\begin{align*}
	{\left(\frac{9917\sqrt{2}}{157}\right) }^2 & = \frac{392.41373}{\sqrt{r}}+3.83379                                                      \\
	r                                          & = {\left(\frac{{\left(\frac{9917\sqrt{2}}{157}\right)}^2-3.83379}{392.41373}\right)}^{-2} \\
	r                                          & \approx 2.42\cdot10^{-3}
	.\end{align*}

The radius of the Sun is about $4.65\cdot 10^{-6}$ AU. Jupiter would need to be very very close to the sun in order to lose its hydrogen.
\end{document}
