\documentclass[../hw10]{subfiles}
\begin{document}
\begin{problem}[3]
Let $C_r$ be the circle of radius  $r$ centered about the origin in the $xz$-plane  and oriented anticlockwise when viewed from the positive $y$-axis.
Suppose $\mathbf{F}\subset \R^3$ is a $C^1$ vector field on the compliment of the  $y$-axis such that $\oint_{C_{1}}\mathbf{F}\cdot d\mathbf{x} = 5$ and $\text{curl}\,\mathbf{F} = \left(\frac{z}{{(x^2 + z^2)}^2},3,\frac{-x}{{(x^2 + z^2)}^2}\right)$.

Compute $\forall r$, $\oint_{C_r} \mathbf{F}\cdot d\mathbf{x}$.
\end{problem}
\begin{proof}
	We already have $\oint_{C_1}\mathbf{F}\cdot d\mathbf{x}$, so we will consider the annular region parametrized by $\mathbf{s}(u,v) = (u\cos{v},0,u\sin{v})$ where $u\in [1,r]$ and $v\in [0,2\pi]$.
	With anticlockwise orientation of the boundary, the normals of this surface must point in positive $y$, so  $\mathbf{s}_u\times \mathbf{s}_v = (0,u,0)$.

	So, using Stokes with $\text{curl}\,\mathbf{F}\cdot d\mathbf{S} = 3u$, we will take the existing integral for $C_1$ and combine with our integral for radius  $1$ to  $r$,  \[
		5+\int_{0}^{2\pi} \int_{1}^{r} 3u \,du \,dv
		= 5 + 6\pi \left[ \frac{u^2}{2} \right]_1^r
		= 5+3\pi (r^2-1)
		.\]

	Note that, if $r<1$, then we will have $5-3\pi(1-r^2)$ because we take away from the unit disk, with surface integral bounded from $u : r \to 1$.
	But, this is actually the same as the above.
\end{proof}
\end{document}
