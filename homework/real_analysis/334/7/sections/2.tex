\documentclass[../hw7]{subfiles}
\begin{document}
\begin{problem}[2]
Let $\mathcal{C}$ be the Cantor set where $\mathcal{C}_0=[0,1]$, $C_{i + 1} = \frac{1}{3}\mathcal{C}_i \cup \left( \frac{1}{3} \mathcal{C}_i + \frac{2}{3}\right)$ (which removes the middle third of each interval in the previous set), and $\mathcal{C}=\bigcap_{i=0}^{\infty} \mathcal{C}_i$.

Prove that the indicator $\chi_C$ is integrable on [0,1].
\end{problem}
\begin{proof}
	We will show that $\mathcal{C}$ is both closed and has measure zero, and then conclude that $\chi_{\mathcal{C}}$ must be integrable as the measure zero set $\mathcal{C}$ will not affect integrability.

	First, since each $\mathcal{C}_i$ is the finite union of closed sets, then each $\mathcal{C}_i$ is closed.
	Then, as the intersection of only closed sets,  $\mathcal{C}$ must also be closed.

	Second, the length of each $\mathcal{C}_k$ is ${\left( \frac{2}{3} \right)}^k$ since one third of the length of every interval is removed at each iteration.
	So, $\forall \epsilon>0,\, \exists K$ such that $\text{len}(\mathcal{C}_K)={\left( \frac{2}{3} \right) }^K<\epsilon$.
	So, $\mathcal{C}_K$ can be covered by an open set with total length less than \epsilon.
	But, $\forall k,\, \mathcal{C}\subset \mathcal{C}_k$ since $\mathcal{C}$ is an intersection of at least $\mathcal{C}_k$,
	so any cover of $\mathcal{C}_k$ will also cover  $\mathcal{C}$.
	Thus, $C$ has measure zero.

	% Since $\mathcal{C}$ has measure zero and is closed, then its boundary must also have measure zero. We also have that $\mathcal{C}$ is bounded since it is a subset of [0,1].
	% Since  $\mathcal{C}$ is bounded and $\partial{\mathcal{C}}$ has measure zero, then $\mathcal{C}$ is Jordan measurable.

	Since $\chi_{\mathcal{C}}$ is bounded on [0,1] and continuously zero on $[0,1]\setminus \mathcal{C}$, then it is integrable on [0,1].
\end{proof}
\end{document}
