\documentclass[../hw9]{subfiles}
\begin{document}
\begin{problem}[5]
Let $\textbf{F} = ((1+3x^3)\exp(x^3 + y^3) + x,\,  3xy^2\exp(x^3 + y^3))$.
\begin{enumerate}[label=(\alph*)]
	\item What is the maximal domain of $\textbf{F}$?
	\item Evaluate $\int_{C_1} \textbf{F} \cdot d \textbf{x}$ where $C_1$ is the vertical line segment from (1,1) to (1,3).
	\item Determine whether $\textbf{F}$ is conservative on the domain from (a).
	      If yes, find a potential function for  $\textbf{F}$.
	      If not, prove that $\textbf{F}$ is not conservative.
	\item Evaluate $\int_{C_2} \textbf{F}\cdot d \textbf{x}$ where $C_2$ is the left semicircle defined by  ${(x-1)}^2 + {(y-2)}^2 = 1,\, x\le 1$ traversed from (1,1) to (1,3).
\end{enumerate}
\end{problem}
\begin{proof}[Proof of a]
	Since each component of $\textbf{F}$ is the product of everywhere continuous functions, then  $\R^2$ is the maximal domain for $\textbf{F}$.
\end{proof}
\begin{proof}[Proof of b]
	Let $\gamma(t)=(1,1 + 2t),\, t\in [0,1]$ parametrize $C_1$.
	Then $\gamma'(t)=(0,2)$.

	So, \[
		\int_{0}^{1} \textbf{F}(\gamma(t))\cdot \gamma'(t) \,dt
		= \int_{0}^{1} 2(3{(1+2t)}^2)\exp(1+{(1+2t)}^3) \,dt
		.\]

	Let $u = 1 + {(1+2t)}^3$ such that $du=2(3{(1+2t)}^2)dt$.
	So, the above becomes \[
		\int^{u(1)}_{u(0)} e^u \,du = {e^u}\big\vert_2^{65}=e^{65}-e^2
		.\]
\end{proof}
\begin{proof}[Proof of c]
	We will show that $\textbf{F}$ is the gradient field of a potential function $f$.

	We will integrate $\textbf{F}_2$ with respect to $y$, as  $f(x,y)=\int \textbf{F}_2\,dy + h(x)$;
	we will recover $h(x)$ from the difference between  $\int \textbf{F}_1\,dx$ and $\int \textbf{F}_2\,dy$.

	So, with $u=x^3 + y^3$, and noting that $x$ is constant when integrating in  $y$,
	\begin{align*}
		\int \textbf{F}_2\,dy & = \int 3xy^2\exp(x^3+y^3)\,dy \\
		                      & = \int xe^u\,du               \\
		                      & = x\exp(x^3+y^3)
		.\end{align*}

	So, $f(x,y) = x\exp(x^3 + y^3) + h(x)$

	Then, letting $h(x)=\frac{x^2}{2}$, we can verify our potential function $f$ by recovering  $\textbf{F}_1$ from  $\frac{\partial f}{\partial x}$.

	Since $\textbf{F}$ is the gradient of a potential function  $f$, it is conservative.
\end{proof}
\begin{proof}[Proof of d]
	Since $\textbf{F}$ is conservative, then  $\textbf{F}$ is path-independent.

	Since $C_1$ and  $C_2$ share respective start and end points, then
	$\int_{C_1}\textbf{F}\cdot d \textbf{x}
		= \int_{C_2}\textbf{F}\cdot d \textbf{x}
		= e^{65}-e^2$.
\end{proof}
\end{document}
